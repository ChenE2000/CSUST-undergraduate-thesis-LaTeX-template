

\song\xiaosi 长沙理工大学(Changsha University of Science \& Technology),简称“CSUST”,位于湖南省长沙市,是国家交通运输部和湖南省人民政府共建高校 ,属于湖南省“国内一流大学建设高校”(A类),入选“中西部高校基础能力建设工程”、教育部“卓越工程师教育培养计划”、教育部“大学生创新性实验计划”,是中国电力高校联盟、绿色交通联盟、中俄交通大学联盟成员之一,是全国毕业生就业典型经验高校,全国深化创新创业教育改革示范高校。

长沙理工大学由原长沙交通学院、长沙电力学院于2003年合并组建。原长沙交通学院的前身是交通部1956年创办的长沙航务工程学校,原长沙电力学院的前身是电力工业部1956年创办的长沙水力发电学校。创办于1956年的湖南省水利水电学校和创办于1958年的湖南省轻工业学校(后更名为湖南轻工业高等专科学校)相继于2001年和2002年并入原长沙电力学院。

截至2022年11月,学校有金盆岭、云塘两个校区,占地面积2980亩,校舍总建筑面积140余万平方米,图书馆纸本藏书360万册;有22个教学学院,1个独立学院、1个继续教育学院,85个本科专业;拥有博士后科研流动站5个,一级学科博士学位授权点8个,二级学科博士学位授权点42个,博士专业学位授权点1个,一级学科硕士学位授权点29个,二级学科硕士学位授权点134个,硕士专业学位授权点18个,5个学科进入ESI全球排名前1\%;有专任教师近2100人,在校生45000余人(含城南学院7000余人),其中硕士、博士研究生8500余人。

\vspace{\baselineskip}
\noindent \sihao\hei 关键词: \song\xiaosi 长沙理工大学;本科毕业论文;模板